\documentclass[a4paper, 11pt]{article}
\usepackage{graphicx}
\usepackage{natbib}
\usepackage{dsfont}
\usepackage[left=3cm,top=3cm,right=3cm]{geometry}

\renewcommand{\topfraction}{0.85}
\renewcommand{\textfraction}{0.1}
\parindent=0cm

\title{Document Title}
\author{Brendon J. Brewer}

\begin{document}
\maketitle

\section{Notation}

\begin{table}[h!]
\begin{center}
\begin{tabular}{|l|c|}
\hline
{\bf Quantity} & {\bf Notation}\\
\hline
Number of particles & $N$\\
Coordinates of system & $\mathbf{q}(t)$\\
Velocities of system  & $\mathbf{v}(t)$\\
PDF for state of system & $f(\mathbf{q}, \mathbf{v}; t)$\\
\hline
\end{tabular}
\end{center}
\end{table}

\section{Example System}

For a system of $N$ particles described in cartesian coordinates, we have:
\begin{eqnarray}
\mathbf{q} &=& \{(x_1, y_1, z_1), (x_2, y_2, z_2), ..., (x_N, y_N, z_N)\}\\
\mathbf{v} &=& \{(v_{x,1}, v_{y,1}, v_{z,1}), (v_{x,2}, v_{y,2}, v_{z,2}), ..., (v_{x,N}, v_{y,N}, v_{z,N})\}\\
\end{eqnarray}
where all of this stuff is a function of time. Let the mass of each particle be
$m$ (all the same mass).

The empirical measure of the positions is the REAL, ACTUAL density function
as a function of three spatial coordinates $(x, y, z)$. Call that $\rho(x,y,z)$.
\begin{eqnarray}
\rho(x, y, z; t) &=& m \sum_{i=1}^N\left[
\delta(x - x_i)\delta(y - y_i)\delta(z - z_i)
\right]
\end{eqnarray}
This is just a function which has spikes on the particle locations.

We don't know the particle locations, we just have a prior PDF $f$ for the
positions and the velocities. The expected value of $\rho$ is:
\begin{eqnarray}
\mathds{E}\left(\rho(x, y, z; t)\right) &=& m
\mathds{E}\left(
\sum_{i=1}^N\left[
\delta(x - x_i)\delta(y - y_i)\delta(z - z_i)
\right]
\right)
\end{eqnarray}
Need to show this gives the mixture of all the marginal distributions for the
positions. Might need to assume exchangeability in $f$, no biggie.

\end{document}

